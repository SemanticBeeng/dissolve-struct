\documentclass[twoside,11pt]{article}


% JMLR Machine Learning Open Source Software papers:
% those are 4 pages plus references
% more details:
% http://www.jmlr.org/mloss/mloss-info.html


% Any additional packages needed should be included after jmlr2e.
% Note that jmlr2e.sty includes epsfig, amssymb, natbib and graphicx,
% and defines many common macros, such as 'proof' and 'example'.
%
% It also sets the bibliographystyle to plainnat; for more information on
% natbib citation styles, see the natbib documentation, a copy of which
% is archived at http://www.jmlr.org/format/natbib.pdf

\usepackage{jmlr2e}
\usepackage{hyperref}
\usepackage{xspace}
\usepackage{bm,color}
\usepackage{mathdots}
\usepackage{subcaption}

\definecolor{darkgreen}{rgb}{0,.4,.2}
\definecolor{darkblue}{rgb}{.1,.2,.6}
\definecolor{brightblue}{rgb}{0,0.6,0.8}
\hypersetup{
  colorlinks=true,
  linkcolor=darkblue,
  citecolor=darkgreen,
  filecolor=darkblue,
  urlcolor=darkblue
}

% Definitions of handy macros can go here

\newcommand{\algname}{\textsc{Dissolve}$\,^{\textsf{\tiny struct}}$\xspace}
\newcommand{\svmstruct}{\textsc{SVM}$\,^{\textsf{\tiny struct}}$\xspace}
\newcommand{\cocoa}{\textsc{CoCoA}\xspace}
\newcommand{\spark}{\textsc{Spark}\xspace}

\newcommand{\dataset}{{\cal D}}
\newcommand{\fracpartial}[2]{\frac{\partial #1}{\partial  #2}}

% Heading arguments are {volume}{year}{pages}{submitted}{published}{author-full-names}

\jmlrheading{1}{2015}{1-??}{1/15}{1/15}{Tribhuvanesh Orekondy, Martin Jaggi and Aurelien Lucchi}

% Short headings should be running head and authors last names

\ShortHeadings{Learning with Mixtures of Trees}{Orekondy, Jaggi and Lucchi}
\firstpageno{1}

\begin{document}

\title{\algname -- A Library for Distributed Structured Prediction}

\author{%
       \name Tribhuvanesh Orekondy \email torekond@student.ethz.ch  \\
       \addr ETH Zurich
       \AND
       \name Martin Jaggi \email jaggi@inf.ethz.ch \\
       \addr ETH Zurich
       \AND
       \name Aurelien Lucchi \email aurelien.lucchi@inf.ethz.ch  \\
       \addr ETH Zurich
       }
       

\editor{t.b.d.}

\maketitle

\begin{abstract}%   <- trailing '%' for backward compatibility of .sty file
This paper describes \algname, a modular and flexible
open source software package for distributed training of structured
prediction models, such as structured SVMs. 
Project website: \href{http://github.com/dalab/dissolve}{github.com/dalab/dissolve}.

We support a broad range of applications, and interfaces to scala, java and python. Our framework is empowered by the fault tolerant \spark computing platform, and automatically adopts to the existing tradeoffs of computation vs communication cost on real world systems. 
The proposed distributed algorithm combines the recent communication efficient \cocoa scheme \citep{Jaggi:2014vi} with the state of the art primal-dual structured prediction solvers \citep{LacosteJulien:2013ue}, and improves further by adding some new ideas for caching oracle answers. 
The framework allows approximate inference, and provides a similar standard interface as \svmstruct for the user.
\end{abstract}

\begin{keywords}
  Structured Prediction, Structured SVM, Distributed Training
\end{keywords}

\section{Introduction}

Bla bla...\\

{\noindent \em Remainder omitted in this sample. See http://www.jmlr.org/papers/ for full paper.}

% Acknowledgements should go at the end, before appendices and references

\acks{...}

% Manual newpage inserted to improve layout of sample file - not
% needed in general before appendices/bibliography.

\newpage

\appendix
\section*{Appendix A.}
\label{app:theorem}

% Note: in this sample, the section number is hard-coded in. Following
% proper LaTeX conventions, it should properly be coded as a reference:

%In this appendix we prove the following theorem from
%Section~\ref{sec:textree-generalization}:

In this appendix we prove the following theorem from
Section~6.2:

\noindent
{\bf Theorem} {\it Let $u,v,w$ be discrete variables such that $v, w$ do
not co-occur with $u$ (i.e., $u\neq0\;\Rightarrow \;v=w=0$ in a given
dataset $\dataset$). Let $N_{v0},N_{w0}$ be the number of data points for
which $v=0, w=0$ respectively, and let $I_{uv},I_{uw}$ be the
respective empirical mutual information values based on the sample
$\dataset$. Then
\[
	N_{v0} \;>\; N_{w0}\;\;\Rightarrow\;\;I_{uv} \;\leq\;I_{uw}
\]
with equality only if $u$ is identically 0.} \hfill\BlackBox

\noindent
{\bf Proof}. We use the notation:
\[
P_v(i) \;=\;\frac{N_v^i}{N},\;\;\;i \neq 0;\;\;\;
P_{v0}\;\equiv\;P_v(0)\; = \;1 - \sum_{i\neq 0}P_v(i).
\]
These values represent the (empirical) probabilities of $v$
taking value $i\neq 0$ and 0 respectively.  Entropies will be denoted
by $H$. We aim to show that $\fracpartial{I_{uv}}{P_{v0}} < 0$....\\

{\noindent \em Remainder omitted in this sample. See http://www.jmlr.org/papers/ for full paper.}


\vskip 0.2in
\bibliography{bibliography}

\end{document}
