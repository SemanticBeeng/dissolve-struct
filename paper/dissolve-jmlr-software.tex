\documentclass[twoside,11pt]{article}


% JMLR Machine Learning Open Source Software papers:
% those are 4 pages plus references
% more details:
% http://www.jmlr.org/mloss/mloss-info.html


% Any additional packages needed should be included after jmlr2e.
% Note that jmlr2e.sty includes epsfig, amssymb, natbib and graphicx,
% and defines many common macros, such as 'proof' and 'example'.
%
% It also sets the bibliographystyle to plainnat; for more information on
% natbib citation styles, see the natbib documentation, a copy of which
% is archived at http://www.jmlr.org/format/natbib.pdf

\usepackage{jmlr2e}
\usepackage{hyperref}
\usepackage{xspace}
\usepackage{bm,color}
\usepackage{mathdots}
\usepackage{subcaption}
\usepackage{amsopn}
\usepackage{amsmath}
\usepackage{multirow}
\usepackage{amssymb}% http://ctan.org/pkg/amssymb
\usepackage{pifont}% http://ctan.org/pkg/pifont

\newcommand{\cmark}{\ding{51}}%
\newcommand{\xmark}{\ding{55}}%

\definecolor{darkgreen}{rgb}{0,.4,.2}
\definecolor{darkblue}{rgb}{.1,.2,.6}
\definecolor{brightblue}{rgb}{0,0.6,0.8}
\hypersetup{
  colorlinks=true,
  linkcolor=darkblue,
  citecolor=darkgreen,
  filecolor=darkblue,
  urlcolor=darkblue
}

% Definitions of handy macros can go here

\newcommand{\algname}{\textsc{Dissolve}$\,^{\textsf{\tiny struct}}$\xspace}
\newcommand{\svmstruct}{\textsc{SVM}$\,^{\textsf{\tiny struct}}$\xspace}
\newcommand{\cocoa}{\textsc{CoCoA}\xspace}
\newcommand{\spark}{\textsc{Spark}\xspace}
\newcommand{\vectornorm}[1]{\left|\left|#1\right|\right|} 

\newcommand{\comment}[1]{}
\newcommand{\dataset}{{\cal D}}
\newcommand{\fracpartial}[2]{\frac{\partial #1}{\partial  #2}}

\newcommand{\av}{\bm{\alpha}}
\newcommand{\xv}{{\bf x}}
\newcommand{\yv}{{\bf y}}

\DeclareMathOperator*{\E}{\rm E}
\newcommand{\simplex}{\Delta}

% general math
\newcommand{\argmax}{\operatornamewithlimits{argmax}}
\newcommand{\argmin}{\operatornamewithlimits{argmin}}
\newcommand{\kroneckerdelta}[2]{\delta_{#2}(#1)}
\newcommand{\realnumbers}{\mathbb{R}}
\newcommand{\R}{\mathbb{R}}
\newcommand{\transpose}{\mathsf{T}}
\newcommand{\norm}[1]{\left\lVert#1\right\rVert}
\newcommand{\floor}[1]{\left\lfloor#1\right\rfloor}
\newcommand{\ceil}[1]{\left\lceil#1\right\rceil}
\newcommand\SetOf[2]{\left\{#1\,\vphantom{#2}\right.\left|\vphantom{#1}\,#2\right\}}

%vectors
\newcommand{\id}{\mathbf{I}} % big i for identity
\newcommand{\ind}{\mathbf{1}} % indicator vectors
\newcommand{\0}{\mathbf{0}} % the origin
\newcommand{\unit}{\mathbf{e}} % unit basis vectors
\newcommand{\one}{\mathbf{1}} % all one vector

\newcommand{\bv}{\bm{b}}

% structured output prediction
\newcommand{\inputvar}{x}
\newcommand{\inputvarv}{\bm{\inputvar}}
\newcommand{\inputdomain}{\mathcal{X}}
\newcommand{\outputvar}{y}
\newcommand{\outputvarv}{\bm{\outputvar}}
\newcommand{\outputdomain}{\mathcal{Y}}
%\newcommand{\hiddenvar}{z}
%\newcommand{\hiddenvarv}{\bm{\hiddenvar}}
%\newcommand{\hiddendomain}{\mathcal{Z}}
\newcommand{\inputoutputdomain}{\mathcal{H}}
\newcommand{\featuremap}{\phi}
\newcommand{\featuremapv}{\bm{\featuremap}}
\newcommand{\featuremapdiff}{\psi}
\newcommand{\featuremapdiffv}{\bm{\featuremapdiff}}
\newcommand{\errorterm}{L}
\newcommand{\weight}{w}
\newcommand{\weightv}{\bm{\weight}}
\newcommand{\wv}{{\bm{\weight}}}
\newcommand{\dualvar}{\alpha}
\newcommand{\dualvarv}{\bm{\alpha}}
\newcommand{\kernelmatrix}{K}
\newcommand{\kernelmatrixv}{\bm{\kernelmatrix}}
\newcommand{\Lmax}{L_{\text{\tiny{max}}}}

% Frank-Wolfe
\newcommand{\sv}{\bm{s}}
\newcommand{\x}{{\bm{x}}} %not to be confused with SVM x
\newcommand{\y}{{\bm{y}}} %not to be confused with SVM y
\newcommand{\e}{{\bm{e}}} % appears in proof of linear convergence...
\newcommand{\stepsize}{\gamma}
\newcommand{\domain}{\mathcal{D}} %TODO: be careful!
%\newcommand{\CfTotal}{C_f^{\text{\tiny{prod}}}}
\newcommand{\CfTotal}{C_{\hspace{-0.1em}f}^\otimes}
\newcommand{\Cf}{C_{\hspace{-0.08em}f}}
\DeclareMathOperator{\approxLin}{\textsc{ApproxLinearMin}}
\DeclareMathOperator{\exactLin}{\textsc{LinearMin}}
\newcommand{\approxArgMin}{\operatornamewithlimits{[approx]\argmin}}
\newcommand{\X}{\mathcal{X}}
\newcommand{\vv}{{\bm{v}}}
\newcommand{\ww}{{\bm{w}}}
\newcommand{\zz}{{\bm{z}}}
\newcommand{\mapprox}{\nu} % multiplicative approximation quality
\newcommand{\addFactor}{\tilde{\gamma}} % constant appearing in the additive approximatoin quality

\newcommand{\note}[1]{{\color{red}#1}}
\newcommand{\ignore}[1]{}


% Heading arguments are {volume}{year}{pages}{submitted}{published}{author-full-names}

\jmlrheading{1}{2015}{1-??}{1/15}{1/15}{Tribhuvanesh Orekondy, Martin Jaggi and Aurelien Lucchi}

% Short headings should be running head and authors last names

\ShortHeadings{Learning with Mixtures of Trees}{Orekondy, Jaggi and Lucchi}
\firstpageno{1}

\begin{document}

\title{\algname -- A Library for Distributed Structured Prediction}

\author{%
       \name Tribhuvanesh Orekondy \email torekond@student.ethz.ch  \\
       \addr ETH Zurich
       \AND
       \name Martin Jaggi \email jaggi@inf.ethz.ch \\
       \addr ETH Zurich
       \AND
       \name Aurelien Lucchi \email aurelien.lucchi@inf.ethz.ch  \\
       \addr ETH Zurich
       }
       

\editor{t.b.d.}

\maketitle

\begin{abstract}%   <- trailing '%' for backward compatibility of .sty file
This paper describes \algname, a modular and flexible
open source software package for distributed training of structured
prediction models, such as structured SVMs. 
Project website: \href{http://github.com/dalab/dissolve}{github.com/dalab/dissolve}.

We support a broad range of applications, and interfaces to scala, java and python. Our framework is empowered by the fault tolerant \spark computing platform, and automatically adopts to the existing tradeoffs of computation vs communication cost on real world systems. 
The proposed distributed algorithm combines the recent communication efficient \cocoa scheme \citep{Jaggi:2014vi} with the state of the art primal-dual structured prediction solvers \citep{LacosteJulien:2013ue}, and improves further by adding some new ideas for caching oracle answers. 
The framework allows approximate inference, and provides a similar standard interface as \svmstruct for the user.
\end{abstract}

\begin{keywords}
  Structured Prediction, Structured SVM, Distributed Training
\end{keywords}

\section{Introduction}

Structured prediction has gained a lot of popularity over the past few years due to its ability to process structured objects such as images or text documents.

The structured support vector machine (SSVM) is a particularly successful variant of this approach that can be optimized in various ways, including a cutting-plane, stochastic gradient descent or a primal-dual scheme such as MARTIN's paper. The latter is especially appropriate for large-scale problems ...

{\noindent \em Remainder omitted in this sample. See http://www.jmlr.org/papers/ for full paper.}

\section{Structured SVM formulation}

A structured model predicts the labeling $\yv$ for a given input $\xv$ by maximizing some score function
$S_\wv:\mathcal{X} \times \mathcal{Y} \rightarrow \mathbb{R}$,
i.e.,
%
\begin{equation}
\label{eq:inference}
\hat{\yv} = \argmax_{\yv \in \mathcal{Y}} S_\wv(Y) = \argmax_{Y \in \mathcal{Y}} \wv^T\Psi(\xv,\yv),
\end{equation}
%
where $\wv$ are the model parameters and $\Psi(\xv,\yv)$ is the \emph{feature map} corresponding to the input $\xv$ and the labeling $\yv$.


Given a set of $n$ training examples $\dataset=\{(\xv_i,\yv_i), \dots, (\xv_n, \yv_n)\}$ where $\xv_i \in \mathcal{X}$ and $\yv_i \in \mathcal{Y}$ is the associated labeling, 
the learning task consists in finding model parameters $\wv$ that achieve low empirical loss subject to some regularization. In other words, we seek
%
\begin{align}
\label{eq:obj}
%\wv^* &= \argmin_\wv \mathcal{L}(\dataset, \wv) \nonumber \\
\wv^* = \argmin_{\wv} \sum_{n=1}^n l(\xv_i, \yv_i, \wv) + R(\wv),
\end{align}
%
where $l$ is the \emph{surrogate loss} function,
a quantity that is usually related to and often upper-bounds the training error, 
and $R(\wv)$ is the regularizer (such as the L2 norm of $\wv$)
that helps prevent overfitting. 
The most common choice of $l$ is the hinge loss, as used in~\cite{Taskar:2003tt,Tsochantaridis:2005ww}, which is defined as
\begin{equation}
\label{eq:hinge-loss}
l(\yv_i, \yv^*_\wv, \wv) = [S_\wv(\yv^*_\wv) + \Delta(\yv_i, \yv^*_\wv) - S_\wv(\yv_i)]_+
\end{equation}


The most popular method to solve problems of the form~\eqref{eq:obj}
is the stochastic subgradient method (SGD)~\citep{Ratliff:2007ti,ShalevShwartz:2010cg}. Although this method has
been quite successful it requires tuning parameters to achieve good
performance (SAY SOMETHING about parallel SGD).We also implemented the
COCOA framework~\cite{Jaggi:2014vi} which we briefly summarized here.


The Lagrange dual of the above $n$-slack-formulation (\ref{eq:svmstruct_nslack_primal}) has $m := \sum_i |\outputdomain_i|$ variables or potential `support vectors'.
Writing $\dualvar_i(\outputvarv)$ for the dual variable associated with the training example $i$ and potential output $\outputvarv \in \outputdomain_i$, the dual problem is given by
\begin{align}
    \label{eq:svmstruct_nslack_dual} % or do we prefer to write it as a maximization?
    \min_{\substack{ \dualvarv\in\R^{m} \\  \dualvarv \geq 0}} \quad  f(\dualvarv) \;:=&  \;\;
    \frac{\lambda}{2}
    \big\| A\dualvarv \big\|^2
    - \bv^T\dualvarv
    \\[-2mm]
    \text{s.t.} \quad &  \;
      \textstyle\sum_{\outputvarv \in \outputdomain_i}  \dualvar_i(\outputvarv) = 1 ~~~\forall i\in[n] \ , \notag 
\end{align}
where the matrix $A\in\R^{d\times m}$ consists of the $m$ columns $A := \SetOf{\frac1{\lambda n} \featuremapdiffv_i(\outputvarv) \in\R^d}{i\in[n],\outputvarv \in \outputdomain_i}$, and the vector $\bv \in \R^m$ is given by 
$\bv:= \left(\frac1n \errorterm_i(\outputvarv) \right)$$_{i\in[n],\outputvarv\in\outputdomain_i}$. %TODO: replace n by N
Given a dual variable vector $\dualvarv$, we can use the Karush-Kuhn-Tucker optimality conditions  to obtain the corresponding primal variables~$
\wv = A\dualvarv  = \sum_{i,\,\outputvarv \in \outputdomain_i} \dualvar_i(\outputvarv)  \frac{\featuremapdiffv_i(\outputvarv)}{\lambda n}
$, see Appendix~\ref{sec:app_duals}.
The gradient of $f$ then takes the simple form $\nabla f(\dualvarv) = \lambda A^TA\dualvarv - \bv = \lambda A^T\wv - \bv$; its \mbox{$(i,\outputvarv)$-th} component is $-\frac{1}{n} H_i(\outputvarv; \wv)$, cf.~\eqref{eq:subproblem_loss_augm}. 
Finally, note that the domain $\domain \subset \R^m$ of~\eqref{eq:svmstruct_nslack_dual} is the product of $n$ probability simplices, $\domain := \simplex_{|\outputdomain_1|}\times\mathellipsis\times\simplex_{|\outputdomain_n|}$.

\comment{
This approach exploits the associated conjugate \emph{dual} problem of \eqref{eq:sdcaPrimal} defined over one dual variable per each example in
the training set.
\begin{equation}
    \label{eq:sdcaDual}
    \max_{\av \in \R^n} \quad \Big[ \ 
    D(\av) := - \frac{\lambda}{2} \norm{ A\av }^2
    - \frac1n \sum_{i=1}^n \ell_i^*(-\alpha_i) \ \Big],
\end{equation}
where $\ell_i^*$ is the conjugate (Fenchel dual) of the loss function
$\ell_i$%
, and the data matrix $A\in\R^{d\times n}$ collects the (normalized) data
examples $A_i := \frac{1}{\lambda n} X^i$ in its columns. The duality comes with the convenient mapping from dual to primal variables
$\wv(\av) := A\av$ as given by the optimality
conditions~\cite{ShalevShwartz:2013wl}.
For any configuration of the dual variables $\av$, we have the duality gap
defined as $P(\wv(\av)) - D(\av)$. This gap is a computable certificate of
the approximation quality to the unknown true optimum $P(\wv^*) = D(\av^*)$,
and therefore serves as a useful stopping criteria as well as a way to automatically set the stepsize. We refer the reader to~\cite{Jaggi:2014vi} for further details.
}

{\bf TODO: Explain main steps}

\section{Software package}

The use of the \algname requires users to implement the following functions:
\begin{itemize}
\item The joint feature map $\Psi(X,Y)$ which encodes the input/output pairs.
\item The structured loss function $\Delta(Y_i,Y)$.
\item A maximization oracle which computes the most violating constraint by solving Eq. X.
\item A prediction function that computes $X$. This operation is usually performed with the maximization oracle.
\end{itemize}

%%%%%%%%%%%%%%%%%%%%%%%%%%%%%%%%%%%%%%%%%%%%%%%%%%%%%%%%%%%%%%%%%%%%%
\begin{table}[h]
\caption{Comparison of structured prediction software packages}
\label{tab:datasets}
   \begin{center}
      \begin{tabular}{l l l c c c c}
       \vspace{.25em}
    {\small\textbf{Package}} & {\small\textbf{Language}} & %
    {\small\textbf{License}} & \small\textbf{Distributed} & \multicolumn{3}{c}{\small\textbf{Models}} \\
    \hline
    & & & & ML & Chain & Graph \\
    \algname & Scala & ? & \checkmark & \xmark & \checkmark & \checkmark \\
    PyStruct & Python & BSD & \checkmark & \xmark & \checkmark & \checkmark \\
	SVM\textsuperscript{struct} & C++ & non-free & \checkmark & \xmark & \xmark & \xmark \\
	Dlib & C++ & boost & \checkmark & \xmark & \checkmark & \checkmark \\
	CRFsuite & C++ & BSD & \checkmark & \checkmark & \checkmark & \xmark \\
      \end{tabular}
   \end{center}\vspace{-2mm}
\end{table}
%%%%%%%%%%%%%%%%%%%%%%%%%%%%%%%%%%%%%%%%%%%%%%%%%%%%%%%%%%%%%%%%%%%%%

TODO: Can we show a simple example like the pystruct paper?

\section{Experiments}

Run experiments on cov and other dataset. Compare to other frameworks...

% Acknowledgements should go at the end, before appendices and references
\acks{We would like to thank Jan Deriu, Bettina Messmer, Thijs Vogels and Ruben Wolff for contributing to the code and discussions to improve readability.}


% Manual newpage inserted to improve layout of sample file - not
% needed in general before appendices/bibliography.

\newpage

\appendix
\section*{Appendix A.}
\label{app:theorem}

% Note: in this sample, the section number is hard-coded in. Following
% proper LaTeX conventions, it should properly be coded as a reference:

%In this appendix we prove the following theorem from
%Section~\ref{sec:textree-generalization}:

In this appendix we prove the following theorem from
Section~6.2:

\noindent
{\bf Theorem} {\it Let $u,v,w$ be discrete variables such that $v, w$ do
not co-occur with $u$ (i.e., $u\neq0\;\Rightarrow \;v=w=0$ in a given
dataset $\dataset$). Let $N_{v0},N_{w0}$ be the number of data points for
which $v=0, w=0$ respectively, and let $I_{uv},I_{uw}$ be the
respective empirical mutual information values based on the sample
$\dataset$. Then
\[
	N_{v0} \;>\; N_{w0}\;\;\Rightarrow\;\;I_{uv} \;\leq\;I_{uw}
\]
with equality only if $u$ is identically 0.} \hfill\BlackBox

\noindent
{\bf Proof}. We use the notation:
\[
P_v(i) \;=\;\frac{N_v^i}{N},\;\;\;i \neq 0;\;\;\;
P_{v0}\;\equiv\;P_v(0)\; = \;1 - \sum_{i\neq 0}P_v(i).
\]
These values represent the (empirical) probabilities of $v$
taking value $i\neq 0$ and 0 respectively.  Entropies will be denoted
by $H$. We aim to show that $\fracpartial{I_{uv}}{P_{v0}} < 0$....\\

{\noindent \em Remainder omitted in this sample. See http://www.jmlr.org/papers/ for full paper.}


\vskip 0.2in
\bibliography{references}

\end{document}
